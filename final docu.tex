%%%%%%%%%%%%%%%%%%%%%%%%%%%%%%%%%%%%%%%%%%%%%%%%
\input{preamble.ltx}

% correct bad hyphenation here; separate each word by a space
\hyphenation{op-tical net-works semi-conduc-tor}

%%%%%%%%%%%%%%%%%%%%%%%%%%%%%%%%%%%%%%%%%%%%%%%%
\begin{document}

\title{Bluetooth Enabled Panic Button for Potential Rape Victims}

\author{
	{\small
		%\renewcommand{\arraystretch}{1.3}
		\begin{tabular}{l l l}
			& \multicolumn{2}{c}{\tiny \textcolor[rgb]{0.9,0.9,0.9}{Participated in \& did the coursework [Y/N]?}} 
			\\ 
			GUEVARRA,~Alnair~M.~(11524855) & Y & \includegraphics[height=5ex] {A_signature} 
			\\ 
			HERNANDEZ,~Roy~Stephen~A.~(11530731)     & Y & %\includegraphics[height=5ex]{JDC_signature}
			\\ 
			LAGMAN,~Maria~Josefa~M.~(11531029)  & Y & \includegraphics[height=5ex]{mama_signature} 
			\\
			MOLINA,~Adam~M.(11539607)  & Y & %\includegraphics[height=5ex]{Capture1}
			\\  		
		\end{tabular}
	}
% for author names; note positions of commas and nonbreaking spaces ( ~ ) LaTeX will not break a structure at a ~ so this keeps an author's name from being broken across two lines.
\thanks{\CrmD\protect\\} %<--Do not delete this \thanks{\CrmD\protect\\} 
 \thanks{Coursework Starting Date: \hspace{1ex} July 13, 2018}
\thanks{Submission Date: \hspace{1ex} August 18, 2018}} 

\markboth{\makebox[\columnwidth]{DIGCOMM Project \hfill}%
\hspace{\columnsep}\makebox[\columnwidth]{\hfill}}%
{} % header

\IEEEpubid{\makebox[\columnwidth]{DIGCOMM - EK \hfill DLSU}%
\hspace{\columnsep}\makebox[\columnwidth]{ \hfill {\tiny\textcolor[rgb]{0.4,0.4,0.4}{\prsDy}}}} % footer

\maketitle % for making the title area


\section{Conspectus}
\label{sec:cnspcts}

\IEEEpubidadjcol % needed in second column of first page if using \IEEEpubid

\subsection{What are the objectives of the coursework?}
\begin{enumerate}
	\item To assess any Digital Communications Technology (DCT) to be used in marginalized sector.
	\item To modify the DCT appropriate to the selected marginalized sector.
	\item To appraise the economic, societal, and environmental implications of the modified DCT.
\end{enumerate}	

\subsection{How does the coursework fit with the course and previously done coursework?}
By:
\begin{enumerate}
	\item Involving the modification of a communication platform between potential rape victims and emergency contacts
	
	\item Using digital communications theory to enhance performance of data transmission for the protection of app users.
	
\end{enumerate}	

\subsection{How were the objectives achieved?}
By:
\begin{enumerate}
	\item Targeting the marginalized sector of the DCT which are potential rape victims.
	\item Modifying the multiple access method used by bluetooth from FH-CDMA to TDMA
	
\end{enumerate}

\subsection{What are the key results and generalizations?}
The key results are:
\begin{enumerate}
	\item To help the master (rape victim) in a Piconet maintain more than one connection simultaneously.

\end{enumerate}

\section{Concepts and Principles}
\label{sec:concps}

\subsection{What are the necessary and relevant concepts and principles for understanding the coursework and for supporting the correct results?}
\begin{enumerate}
	\item Mastery of how Bluetooth works and its specifications
	\item Knowledge about different techniques used in Multiplexing methods
	\item Clear understanding of Frequency Hopping Spread Spectrum (FHSS)
	\item Basic concepts about bluetooth connections
	
\end{enumerate}

\subsection{How does any new component, not covered in  previous coursework, function?}
By:
\begin{enumerate}
	\item Overlaying Time Division Multiplexing (TDM) on bluetooth connections. 
	
	\item Incorporating frequency hopping with TDM for simultaneous data transmission
	
	\item Synchronizing a master device with the other two devices through the use of frequency hopping.
	
\end{enumerate}


\subsection{What figures, equations, and/or tables could support your answers in Sec. 2.1 and Sec.2.2?}
\begin{enumerate}
	\item The figure below shows an example of a wireless connection between a single or multiple devices using bluetooth
	\begin{figure}[hp]
		\centering
		\captionsetup{justification=centering,margin=2cm}
		\includegraphics[height=25ex]{piconet}
		\caption{Piconet}
	\end{figure}
	\item The figure below shows a block diagram implementation for frequency hopping code.
	\begin{figure}[ht]
		\centering
		\captionsetup{justification=centering,margin=2cm}
		\includegraphics[height=30ex]{fhss}
		\caption{Block Diagram of FHSS}
	\end{figure}
\end{enumerate}

\subsection{Did you cite more than two publications in your answers in Sec. 2.1. and 2.2}
Yes
	
\subsection{Did you cite any online source in your answers in Sec.2.1 and Sec.2.2?}
Yes.


\section{Methodology}

\subsection{How does your implementation in Sec.~\ref{sec:implem} achieve the objectives?}
By:
\begin{enumerate}
	\item Simulating the performance difference between CDMA and TDMA
\end{enumerate}

\subsection{Why does your implementation in Sec.~\ref{sec:implem} achieve the objectives?}
Because:
\begin{enumerate}
	\item The application of TDMA simplifies the point to multiple-point connections
	
	\item Multiple-point connections can help the rape victim reach out to a lot of contacts simultaneously
	
\end{enumerate}

\subsection{How does your evaluation in Sec.~\ref{sec:eval} achieve the objectives?}
By:
\begin{enumerate}
	\item Up to two lines per item.
	\item Up to two lines per item.
\end{enumerate}

\subsection{Why does your evaluation in Sec.~\ref{sec:eval}  achieve the objectives?}
Because:
\begin{enumerate}
	\item Up to two lines per item.
	\item Up to two lines per item.
\end{enumerate}



\subsection{Implementation}
\label{sec:implem}

Rule of thumb: Implementation is how you made your  work; (keywords: implemented, created, made, soldered, programmed, etc.).


\subsubsection{What were the materials used?}
If the presentation would be better and necessary, tabulate your answers here instead of enumeration.
\begin{enumerate}
	\item Bluetooth Module
	\item Arduino
	\item Push Button
\end{enumerate}


\subsubsection{What is the summary of the processes used to make the coursework?}

\begin{figure}[hp]
	\centering
	\captionsetup{justification=centering,margin=2cm}
	\includegraphics[height=30ex]{bluetoothsimulink}
	\caption{An example of bluetooth module in SIMULINK to be used for DCT}
\end{figure}

\begin{enumerate}
	\item Overlaying of bluetooth protocol to hardware device.
	\item Master device shares different hopping code for each slave devices.
	\item Each slave devices are independent of each other.
\end{enumerate}

\begin{table}[!b]
	\caption{Pseudocode for the calculation of $y = x^n$}
	\label{tab:calcxn}	
	\centering
	{\footnotesize
		\begin{tabular}{lll}
			\hline
			\hline
			{\bfseries Input(s):} & & \\
			$n$ & : & $n$th power; $n \in \mathbb{Z}^{+}$ \\
			$x$ & : & base value; $x \in \mathbb{R}^{+}$ \\
			\hline
			{\bfseries Output(s):} & & \\
			$y$ & : & result; $y \in \mathbb{R}^{+}$  \\
			\hline
			\hline
			\\
		\end{tabular}
	}
	\begin{algorithmic}[1]
		{\footnotesize
			\REQUIRE $n \geq 0 \vee x \neq 0$
			\ENSURE $y = x^n$
			\STATE $y \Leftarrow 1$
			\IF{$n < 0$}
			\STATE $X \Leftarrow 1 / x$
			\STATE $N \Leftarrow -n$
			\ELSE
			\STATE $X \Leftarrow x$
			\STATE $N \Leftarrow n$
			\ENDIF
			\WHILE{$N \neq 0$}
			\IF{$N$ is even}
			\STATE $X \Leftarrow X \times X$
			\STATE $N \Leftarrow N / 2$
			\ELSE[$N$ is odd]
			\STATE $y \Leftarrow y \times X$
			\STATE $N \Leftarrow N - 1$
			\ENDIF
			\ENDWHILE
		}
	\end{algorithmic}
\end{table}







\subsection{Evaluation}
\label{sec:eval}

Rule of thumb: Evaluation is how you tested your  work for correctness; (keywords: measured, tested, compared, simulated, etc.).

\subsubsection{What were your procedures for evaluating the correct outcome of your coursework?}
\begin{enumerate}
	\item Up to two lines per item.
	\item Up to two lines per item.
\end{enumerate}

\subsubsection{What quantities were gathered and how have you obtained them for testing the veracity of your results?}
\begin{enumerate}
	\item Up to two lines per item.
	\item Up to two lines per item.
\end{enumerate}


\section{Results and Discussions}

\subsection{How do the results achieve the objectives?}
By:
\begin{enumerate}
	\item Showing that TDMA was more efficient compared to CDMA. With this, the modified DCT would highly benefit the marginalized sector.
	\item Proving that TDM and FDM overlays protocol are beneficial to helping the device to be able to connect to more than one data connection to other devices simultaneously
	\item Knowing that using bluetooth modules with modified DCT is better than using other modules such as GSM/GPRS because the marginalized sectors are able to get more benefits with less expenses.
	
\end{enumerate}

\subsection{Why do the results achieve the objectives?}
Because:
\begin{enumerate}
	\item The results achieve the objectives by means of economical because the marginalized sector will consume less on the device due to the cheaper module that will be used; 
	
	\item Bluetooth is environmental because it does not release any ionizing radiation, hence being harmless as compared to other radiations that is used by other modules;
	
	\item In addition, the project device overall is societal because of its objective to prevent harm that may be inflicted by rape victims.
	
\end{enumerate}

\subsection{Are all you results correct  in accordance to what you described in Sec.~\ref{sec:eval} evaluation process? Why?} 
Yes, because:
\begin{enumerate}
	\item the researchers were able to overlay TDM with FHSS and showed a simulation of it.
	\item A full duplex system between a master and slave device was made using SIMULINK
\end{enumerate}

\subsection{What}

\begin{enumerate}		
	\item
		\begin{figure}[ht]
			\centering
			\captionsetup{justification=centering,margin=2cm}
			\includegraphics[height=19ex]{timingdiagram}
			\caption{Timing Diagram}
		\end{figure}
	dfdfdf
	\item
	\begin{figure}[ht]
	\centering
	\captionsetup{justification=centering,margin=2cm}
	\includegraphics[height=47ex]{spectrumanal}
	\caption{Spectrum Analyzer}
	\end{figure}
dfdf
	\item 
dfd
	\begin{figure}[ht]
	\centering
	\captionsetup{justification=centering,margin=2cm}
	\includegraphics[height=18ex]{hopping1}
	\caption{Hopping Spectrogram output for connection 1}
	\end{figure}

	\item 
	\begin{figure}[ht]
	\centering
	\captionsetup{justification=centering,margin=2cm}
	\includegraphics[height=18ex]{hopping2}
	\caption{Hopping Spectrogram output for connection 2}
	\end{figure}
dfd
\end{enumerate}

\subsection{Did you cite more than two publications in your answers above (yes/no)?}
Yes.	



\section{Conclusions}
\label{sec:conc}

\subsection{What are the main points that should be known, remembered, and learned about the coursework?}
\begin{enumerate}
	\item Up to two lines per item.
	\item Up to two lines per item.
\end{enumerate}

\subsection{What are the gists of the inferences drawn from your results?}
\begin{enumerate}
	\item Up to two lines per item.
	\item Up to two lines per item.
\end{enumerate}

\subsection{Briefly, what are your comments on (1)~your results, and  (2)~future coursework if any?}
\begin{enumerate}
	\item Up to two lines per item.
	\item Up to two lines per item.
\end{enumerate}	


\bibliographystyle{IEEEtr}
\bibliography{references} % references section

\newpage
\begin{figure*}[!t]
	\includegraphics[width=\textwidth]{rubric} 
\end{figure*}
\cleardoublepage

\end{document}